%!TEX TS-program = xelatex
% vim: set fenc=utf-8

% -*- coding: UTF-8; -*-
%!TEX encoding = UTF-8
% !TeX program = xelatex
\documentclass[twoside]{CUGCSthesis_EN}


% 隐藏链接


\title{Your Title} %论文题目
\author{Your Name} %作者姓名
\date{May 15, 2024} %日期
\school{School of Computer Science} %院系名称
\classnum{\begin{tabular}[t]{@{}l@{}}Computer Science and Technology \\ (Big Data)\end{tabular}}
\stunum {xxxxxxxxxxx} %学号
\instructorone{Prof.xxxx} %指导教师1姓名
%\instructoronelevel{副教授}

\usepackage{float}
\usepackage{makecell}
\usepackage{multicol}
\usepackage{multirow}
% \usepackage[backref]{hyperref}
\usepackage[hypertexnames=false]{hyperref}
\hypersetup{
    bookmarks=false, % Prevent the creation of bookmarks in the PDF
    colorlinks=false, % Disable colored links
    pdfborder={0 0 0}, % Remove border around links
    hidelinks
}
\hypersetup{
hidelinks
}

\usepackage{color}


\begin{document}
	\maketitle
	\makestatement\

	\begin{cnabstract}{关键词;关键词;关键词;关键词}
 	\pagestyle{empyt}
		摘要
	\end{cnabstract}
	
	\begin{enabstract}{keyword1; keyword2; keyword3; keyword4}
		Abstract content
	\end{enabstract}
	
	\makeToc
	
	%---------------------------------------------开始正文---------------------------------------------
	
	%--------------------------------------------- 第一章 ---------------------------------------------
	\begin{spacing}{2}
		\section{Introduction}
	\end{spacing}

	\subsection{Background}
	Dissertations range from more than 40 pages to more than 200 pages. If such a long paper is written in word, there will be many problems.
	Many universities in foreign countries have their theses in \LaTeX templates, and most universities recommend that graduates write their theses in \LaTeX.
	Although most universities in China do not recommend this, there are already many university graduates who make their own latex templates and 
	share them on the Internet.

	For the academic dissertation of China University of Geosciences (Wuhan), although there is no official latex template, many senior students 
	have developed Chinese latex templates for academic dissertations, the English version is still missing.

	\subsection{Instruction}
	This template is based on the ThuThesis latex template, CUGThesis template and the standard for writing English bachelor's degree thesis of China 
	University of Geosciences (Wuhan). It is mainly used for the English bachelor's degree thesis of the School of Computer Science, CUG.

	And you can Refer to "一份(不太)简短的 \LaTeX 2ε 介绍" for more latex writing methods.


	
	
	%------------------------------------------- 第二章 ---------------------------------------------
	\begin{spacing}{2}
		\section{Template use method}
	\end{spacing}
	
		\subsection{Figure}

			\subsubsection{Insert a single graph}

			Insert a single picture.

			\begin{figure}[H]
				\centering
				\includegraphics[scale=0.5]{Figures/example.jpg}
				\caption{Example1}
				\label{Fig:example}
			\end{figure}

			\subsubsection{Insert multiple graphs}

			It is often used to place multiple diagrams in a single float. The following is the schematic code, and the effect is roughly shown 
			in Figure \ref{fig:figure_3col}.

			\begin{verbatim}
				\begin{figure}[htbp]
					\centering
					\includegraphics[width=...]{...}
					\qquad
					\includegraphics[width=...]{...} \\[...pt]
					\includegraphics[width=...]{...}
					\caption{...}
				\end{figure}
			\end{verbatim}

			\begin{figure} [H]
				\centering%
				\subcaptionbox{logo1\label{fig:cug1} }  
				{\includegraphics[height=0.45\textwidth]{Figures/CUG_Logo1}} 
				\hspace{0.01\textwidth}  
				\subcaptionbox{logo2\label{fig:cug2} }  
				{\includegraphics[height=0.45\textwidth]{Figures/CUG_Logo2}} 
				\caption{Example2} 
				\label{fig:figure_3col} 
			\end{figure} 
		
	\subsection{Table or Item}

	The most basic tabular usage of tabular tables is as follows and the effect is roughly shown in Tab \ref{example:Tab}:
	
	\begin{verbatim}
		\begin{tabular}[⟨align⟩]{⟨column-spec⟩}
			⟨item1⟩ & ⟨item2⟩ & … \\
			\hline
			⟨item1⟩ & ⟨item2⟩ & … \\
		\end{tabular}
	\end{verbatim}

	
	\& is used to separate cells;  {\textbackslash \textbackslash} is used to wrap a line; {\textbackslash}\textit{hline} is used to draw horizontal lines between rows.

	\begin{table}[H]
		\centering
		\caption{Example:Tab}
		\label{example:Tab}
		\scalebox{1.0}{
			\begin{tabular}{ccc}
				\toprule  %添加表格头部粗线
				name & num & gender\\
				\midrule  %添加表格中横线
				Steve Jobs& 001& Male\\
				Bill Gates& 002& Female\\
				\hline \hline
			\end{tabular}
		}
		\vspace{10mm}
	\end{table}

	\LaTeX provides the basic ordered and unordered list environments enumerate and itemize, both of which are used similarly 
	Label each list item with {\textbackslash}\textit{item}. The enumerate environment automatically numbers list items.

	\begin{verbatim}
		\begin{enumerate}
			\item …
			\item …
		\end{enumerate}
		
	\end{verbatim}

	\begin{enumerate}
		\item item1
		\item item2
	\end{enumerate}

	\subsection{Reference}

	Cross-reference is one of the powerful automatic typesetting features of \LaTeX. In places that can be cross-referenced, such as chapters, public 
	Formula, chart, theorem, etc. Use {\textbackslash}\textit{label} command:

	You can then use the {\textbackslash}\textit{ref}\ or {\textbackslash}\textit{pageref} commands elsewhere to generate cross-referenced numbers 
	and page numbers, respectively:

	Reference a figure \ref{fig:cug1}

	Reference a table \ref{example:Tab}

	If you want to cite a reference, you1 need to format references into mybib.bib and cite it \cite{1}

	\subsection{Equation}

	Equations are created using the traditional equation environment:

	\begin{equation}
		\label{eqn_example}
		x = \sum\limits_{i=0}^{z} 2^{i}Q
	\end{equation}

	A multiline equation
	\begin{eqnarray}
		Z&{}={}&x_1 + x_2 + x_3 + x_4 + x_5 + x_6\nonumber\\
		&&+a + b\\
		&&+{}a + b\\
		&&{}+a + b\\
		&&{+}\:a + b
	\end{eqnarray}
	
	%--------------------------------------------- 第三章 ---------------------------------------------
	\begin{spacing}{2}
		\section{Conclusion}
	\end{spacing}
	
	Conclusion of the article
	
	%---------------------------------------------  致谢  ---------------------------------------------
	\begin{spacing}{2}
		\section*{Acknowledgement}
	\end{spacing}
	\phantomsection
	\addcontentsline{toc}{section}{Acknowledgement}
	
	Content of acknowledgement
	
	\clearpage
	%---------------------------------------------参考文献---------------------------------------------
	\phantomsection
	\addcontentsline{toc}{section}{References}
	\bibliography{Bibs/mybib}
	
	
\end{document}
